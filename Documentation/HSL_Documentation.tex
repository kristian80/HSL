\documentclass[10pt,a4]{scrartcl}
\usepackage[10pt]{extsizes}
\usepackage{geometry}
\usepackage[pdftex]{graphicx}
\usepackage{epstopdf}
\usepackage[hang,bf]{caption}       %Fuer die Unterschriften
\usepackage[ansinew]{inputenc}      %Fuer deutsche Sonderzeichen!!!
\usepackage{multirow}
%\usepackage{a4wide}
\usepackage[english]{babel}
\usepackage{fancyhdr}
\usepackage{fancyvrb}               %Fuer verbatim style
\usepackage{multirow}
\usepackage{lscape}                 %um Tabellen und usw. im Querformat darzustellen
\usepackage{amssymb}
\usepackage{epsfig}
\usepackage{graphicx}
\usepackage{amsmath}
\usepackage{subfigure}
%\usepackage{times}
%\usepackage[T1]{fontenc}
%\usepackage{helvet}
\usepackage{arev}
%\usepackage{txfonts}
\usepackage{tabularx}
\usepackage{hyperref}
\usepackage[super]{nth}
%\usepackage{setspace}
%\usepackage{threeparttable}

%\setlength{\topmargin}{-2cm}
%\setlength{\oddsidemargin}{0cm}
%\setlength{\evensidemargin}{0cm}
\geometry{a4paper,left=15mm,right=15mm, top=1.3cm, bottom=1.5cm}
\pagestyle{empty}

\pagestyle{fancy}
\fancyhf{}
\fancyfoot[R]{\thepage}
\fancyfoot[L]{}
\renewcommand{\headrulewidth}{0pt}
\renewcommand{\footrulewidth}{0.4pt}

\newcommand{\spa}{0.5mm}
\newcommand{\spb}{3mm}
\newcommand{\taba}{3cm}

\begin{document}

\title{Helicopter Sling Load}
\subtitle{External Load Simulation for X-Plane}
\date{}
\maketitle


\section{Introduction}

Helicopter Sling Load (HSL) is a X-Plane 11 Plugin for the physical simulation of external loads. The purpose of this plugin is to provide an external load simulation that can be fully configured by \nth{3} party plugins. While it is possible for users to fly loads just with HSL, this is not the primary goal.\\

X-Plane is an ever changing environment and therefore plugins need to be maintained on a regular basis to keep working with the current versions. To make sure this plugin can survive even in a case where I cannot actively maintain it, I made it open source. All source code is available at https://github.com/kristian80/HSL and published using the GPL v3 license.\\

\section{Installation}

Extract the zip  files and copy the HSL folder into your "X-Plane\textbackslash Resources\textbackslash Plugins" folder.

\section{How to use this plugin}

Using the GUI you can place objects in front of the helicopter and connect them with the rope. Then you can take off and release the load where you want to. The cargo will continue the physics computation even after release, so it will glide on the ground if you where not slow enough and it will drop if you release it mid flight. If you exceed the limitations of the rope, it will rupture and the cargo will fall down.\\

Plugins may place the cargo at any coordinates in the world, by accessing the corresponding datarefs and commands.\\

When connecting the cargo with a rope that is too short, the winch will extend it further. This way you do not have to worry about the helicopter being soaked towards the cargo.\\

You can use any objects within the X-Plane objects library as a cargo object. For this you need the library path of the objects. A simple method for finding nice objects is to download WED and create a dummy scenery. When opening the scenery you will be able to browse all installed X-Plane libraries, including all the custom scenery you have installed. Just hit the Update Objects button when finished. In case of an error, the plugin will be disabled. Be aware that the GUI is very sensitive to keyboard input and hitting CTRL+V often pastes the text twice.\\

\section{How it works}

The computation of a sling load is based on differential equations that cannot be solved analytically. Hence, HSL uses the corresponding difference equation for the numeric approximation. The accuracy of this numeric approximation is depending on the step size. If the step size would be infinitesimal small, the result would be precise.\\

As it turned out, for small loads the x-plane flight loop time resolution is on the limit even with four flight loops per frame and 100 fps. Furthermore, even small stutters resulted in unrealistic object behavior. Therefore, I have decoupled the physics computation from the x-plane main thread and all the computations run asynchronously in a separate thread. On each flight loop the information about the helicopter position, speed and acceleration is updated. Using this information the plugin computes a virtual helicopter that continues to fly smoothly, even if your x-plane is currently stuttering. This way, I can use far smaller time steps for the computation than the flight loop would allow, resulting in an appropriate accuracy of the numerical approximation.\\

To suit high and low performance computers the plugin supports two modes for the computation:\\

\textbf{High Performance Disabled:} Recommended for CPUs with less cores. The physics thread uses the sleep command after each computation. The x-plane thread goes to sleep whenever waiting for the physics thread. This way the physics thread is prompting the Windows scheduler to perform other tasks in-between computations and resources are released as soon as one thread is waiting for the other one. As the time for the physics computation is magnitudes smaller than the minimum sleep time, the X-Plane flight loop will rarely have to wait for a physics computation to finish.\\

\textbf{High Performance Enabled:} Recommended for CPUs with lots of cores. Both, the physics thread and the x-plane flight loop will wait for the Mutex in a polling loop. If both run on separate cores, this will minimize the step size of the physics computation.
Just try what gives the best result. The Windows scheduler is able to handle high CPU loads, so even with a low core count the High Performance mode might give better results.\\

For fire fighting the drops released by the bambi bucket are also computed in a separate thread. This thread runs synchronously to the flight loop thread, without blocking. If you have too many drops to be computed in-between flight loops, the drops will simply slow down. For performance reasons the number of drops that is drawn is limited, but they will still be considered for extinguishing fires.\\

\subsection{What's included}

\begin{itemize}
\item Gravity:
\begin{itemize}
\item Using the X-Plane dataref. Yes, it should also work on Mars.
\item Rope forces: The rope is considered an single mass oscillator.
\item Stretch forces.
\item Damping forces according to the speed the rope is stretching.
\end{itemize}
\item Air drag: The object is considered cuboid.
\begin{itemize}
\item Object speed plus wind.
\item Air density depending on altitude and weather, using the X-Plane dataref.
\item CW value for each side of the cuboid.
\item Surface of each side.
\item Roll and pitch of the object are considered.
\end{itemize}
\item Water drag:
\begin{itemize}
\item Same as the Air drag, except that the object is always upright as it hits the ground.
\item Air and water drag are weighted according to the water line. This means that the wind might move a floating object.
\end{itemize}
\item Water displacement:
\begin{itemize}
\item Upward forces by the water displacement. If the object is lighter than water, it will swim.
\item Bambi buckets don't swim and fill with water instantly, according to the water line.
\end{itemize}
\item Friction:
\begin{itemize}
\item Static friction when the object is still on the solid ground.
\item Glide friction when the object is moving on the solid ground.
\item Damping by the operator: When lifting an object, the operators on winch or on the hook will usually try to stabilize the load. Otherwise the short rope would start hazardous oscillations.
\item Rope length where the operators start the stabilization.
\item Maximum force the operator may apply .
\end{itemize}
\item For 3rd party plugins:
\begin{itemize}
\item Plugin may apply additional forces (e.g. to simulate ground operator).
\item Plugin may rotate the cargo.
\end{itemize}
\item Forces on the Helicopter:
\begin{itemize}
\item Forces of the rope.
\item Momentum that is caused by the positioning of the winch.
\end{itemize}
\end{itemize}

\subsection{What's missing}

\begin{itemize}
\item Cargo rotation (except rotation by external plugin)
\item Rotor downwash
\item Ground impact:
\begin{itemize}
\item When hitting the ground, the vertical speed is put to zero. No bouncing off the ground.
\item When hitting the ground, the object is always upright. No flipping over.
\end{itemize}
\item There is no momentum on the load. All forces apply in the CoG, which is considered at the rope mounting point.
\item Bambi bucket limited water flow when dropped into water
\item Terrain steepness is not considered. For HSL the terrain is always flat.
\item Cargo maximum forces. Cargo never breaks, no matter how hard it hits the ground.
\item Full rope simulation, currently the rope is always straight.
\item Fire temperature increase of the surroundings
\item Fire turbulence simulation
\item Fire in replay
\end{itemize}

\section{Commands}

Commands can be bound either to a hotkey or joystick button, or they can be called by \nth{3} party plugins using the X-Plane SDK.\\
\\
\begin{tabular}{| l | l |} \hline
\textbf{Command} & \textbf{Description}\\ \hline
\texttt{HSL/Winch\_Up} & Move the winch up\\ \hline
\texttt{HSL/Winch\_Down} & Move the winch down\\ \hline
\texttt{HSL/Winch\_Stop} & Stop the winch\\ \hline
\texttt{HSL/Sling\_Enable} & Enable slingline\\ \hline
\texttt{HSL/Sling\_Disable} & Disable slingline\\ \hline
\texttt{HSL/Sling\_Reset} & Reset slingline\\ \hline
\texttt{HSL/Load\_Connect} & Connect the load\\ \hline
\texttt{HSL/Load\_Release} & Release the load\\ \hline
\texttt{HSL/Load\_On\_Ground} & Place the load in front of the aircraft\\ \hline
\texttt{HSL/Load\_On\_Coordinates} & Place the load at given coordinates\\ \hline
\texttt{HSL/ToogleControlWindow} & Toggle Control Window\\ \hline
\texttt{HSL/UpdateObjects} & Update Objects\\ \hline
\texttt{HSL/Fire\_On\_Ground} & Place fire in front of the aircraft\\ \hline
\texttt{HSL/Fire\_On\_Coordinates} & Place fire at given coordinates\\ \hline
\texttt{HSL/Bucket\_Release} & Release Water in Bambi Bucket\\ \hline
\end{tabular}

\section{DataRefs}


\subsection{Writeable Values}
\subsubsection{Object Paths}

\begin{tabularx}{\linewidth}{| l | l | X |} \hline
\textbf{Type} & \textbf{DataRef Name}& \textbf{Description}\\ \hline
String (dynamic) & \texttt{HSL/WinchObjectPath} & Library or physical path (relative from XP folder) of the winch object. Library is tried first. Send command \texttt{HSL/UpdateObjects} to reload objects.\\ \hline
String (dynamic) & \texttt{HSL/RopeObjectPath} &  Library or physical path (relative from XP folder) of the rope objects. Rope objects are not rotated, use spherical objects only. Library is tried first. Send command \texttt{HSL/UpdateObjects} to reload objects.\\ \hline
String (dynamic) & \texttt{HSL/HookObjectPath} &  Library or physical path (relative from XP folder) of the hook object. Library is tried first. Send command \texttt{HSL/UpdateObjects} to reload objects.\\ \hline
String (dynamic) & \texttt{HSL/CargoObjectPath} & Library or physical path (relative from XP folder)  of the cargo object. Library is tried first. Send command \texttt{HSL/UpdateObjects} to reload objects.\\ \hline
\end{tabularx}
\subsubsection{Winch}
\begin{tabularx}{\linewidth}{| l | l | X |} \hline
\textbf{Type} & \textbf{DataRef Name}& \textbf{Description}\\ \hline
Float Array[3] & \texttt{HSL/Winch/VectorWinchPosition} & Position of the winch in aircraft coordinates.\\ \hline
Double & \texttt{HSL/Winch/WinchSpeed} &  Up/Down speed of the winch [m/s]\\ \hline
\end{tabularx}
\subsubsection{Rope}
\begin{tabularx}{\linewidth}{| l | l | X |} \hline
\textbf{Type} & \textbf{DataRef Name}& \textbf{Description}\\ \hline
Double & \texttt{HSL/Rope/RopeLengthStart} & Length of the rope upon plugin start or reset\\ \hline
Double & \texttt{HSL/Rope/RopeLengthNormal} & Deployed length of the rope, without expansion\\ \hline
Double & \texttt{HSL/Rope/RopeDamping} & Damping factor of the rope. Has to be between zero (undamped) and one (critical damping).\\ \hline
Double & \texttt{HSL/Rope/RopeK} & Force applied when the rope is expanded to twice its normal size [N].\\ \hline
Double & \texttt{HSL/Rope/RuptureForce} & Force that causes the rope to rupture.\\ \hline
Double & \texttt{HSL/Rope/MaxAccRopeFactor} & Not used anymore.\\ \hline
Double & \texttt{HSL/Rope/OperatorForce} & Force of the operator to dampen the oscillation.\\ \hline
Double & \texttt{HSL/Rope/OperatorLength} & Rope length when the operator start dampening the oscillation.\\ \hline
\end{tabularx}
\subsubsection{Hook}
\begin{tabularx}{\linewidth}{| l | l | X |} \hline
\textbf{Type} & \textbf{DataRef Name}& \textbf{Description}\\ \hline

Double & \texttt{HSL/Hook/Height} & Height of the hook object when drawing the hook on the ground [m]. Only for visualization.\\ \hline
Double & \texttt{HSL/Hook/Mass} & Mass of the hook [kg]\\ \hline
Float Array[3] & \texttt{HSL/Hook/Size} & Dimensions of the hook: Depth/Width/Height [m]. Only for physics, not for visualization.\\ \hline
Float Array[3] & \texttt{HSL/Hook/Rotation} & Rotation of the hook [deg]. Only for visualization.\\ \hline
Float Array[3] & \texttt{HSL/Hook/CW} & Aerial resistance of the hook surfaces: Front/Side/Top. \\ \hline
Double & \texttt{HSL/Hook/FrictionGlide} &  Glide friction of the hook, when on solid ground. Relative value [N/N]: Horizontal resistance force resulting from vertical pressure force.\\ \hline
Double & \texttt{HSL/Hook/FrictionStatic} & Static friction of the hook, when on solid ground. Relative value [N/N]: Maximum horizontal resistance force resulting from vertical pressure force.\\ \hline

Float Array[3] & \texttt{HSL/Hook/ExternalForce} & External forces applied on the hook (e.g. by an operator) [N].\\ \hline
\end{tabularx}
\subsubsection{Cargo}
\begin{tabularx}{\linewidth}{| l | l | X |} \hline
\textbf{Type} & \textbf{DataRef Name}& \textbf{Description}\\ \hline
Double & \texttt{HSL/Cargo/SetLatitude} & Latitude value when placing a new cargo at coordinates. Use for command \texttt{HSL/Load\_On\_Coordinates}.\\ \hline
Double & \texttt{HSL/Cargo/SetLongitude} & Longitude value when placing a new cargo at coordinates. Use for command \texttt{HSL/Load\_On\_Coordinates}.\\ \hline

Double & \texttt{HSL/Cargo/Height} & Height of the cargo object when drawing the cargo on the ground [m].  Only for visualization.\\ \hline
Float Array[3] & \texttt{HSL/Cargo/RopeOffset} & Offset of rope in relation to the origin of the cargo object [m]. Only for visualization.\\ \hline
Float Array[3] & \texttt{HSL/Cargo/Rotation} & Rotation of the cargo [deg]. Only for visualization.\\ \hline
Double & \texttt{HSL/Cargo/Mass} &  Mass of the cargo [kg]\\ \hline
Float Array[3] & \texttt{HSL/Cargo/Size} & Dimensions of the cargo: Depth/Width/Height [m]. Only for physics, not for visualization.\\ \hline
Float Array[3] & \texttt{HSL/Cargo/CWFront} & Aerial resistance of the cargo surfaces: Front/Side/Top.\\ \hline
Double & \texttt{HSL/Cargo/FrictionGlide} & Glide friction of the cargo, when on solid ground. Relative value [N/N]: Horizontal resistance force resulting from vertical pressure force.\\ \hline
Double & \texttt{HSL/Cargo/FrictionStatic} & Static friction of the cargo, when on solid ground. Relative value [N/N]: Maximum horizontal resistance force resulting from vertical pressure force.\\ \hline

Integer & \texttt{HSL/Cargo/IsBambiBucket} & Tells the plugin if the cargo is a bambi bucket, either 0 or 1. Bambi buckets don't swim and fill up their volume when lowered into water.\\ \hline
Integer & \texttt{HSL/Cargo/BambiBucketReleaseWater} & Tells the plugin to release the water in the bambi bucket (either 0 or 1).\\ \hline

Float Array[3] & \texttt{HSL/Cargo/ExternalForce} & External forces applied on the cargo(e.g. by an operator) [N].\\ \hline
\end{tabularx}
\subsubsection{Fire}
\begin{tabularx}{\linewidth}{| l | l | X |} \hline
\textbf{Type} & \textbf{DataRef Name}& \textbf{Description}\\ \hline
String (dynamic) & \texttt{HSL/Fire/FireAircraftPath} & Physical path (relative from XP folder) of the AI aircraft that is used to draw the fire.\\ \hline
Double & \texttt{HSL/Fire/WaterRadius} & Within this radius, water drops are counting towards extinguishing the fire [m].\\ \hline

Double & \texttt{HSL/Fire/StrengthStart} & Start value for water required to extinguish the fire [kg].\\ \hline
Double & \texttt{HSL/Fire/StrengthMax} & Maximum value for water required to extinguish the fire [kg].\\ \hline
Double & \texttt{HSL/Fire/StrengthIncrease} & Fire increasing over time,water weight increase per second [kg/s].\\ \hline


Double & \texttt{HSL/Fire/SetLatitude} & Latitude value when placing a new fire at coordinates. Use for command \texttt{HSL/Fire\_On\_Coordinates}.\\ \hline
Double & \texttt{HSL/Fire/SetLongitude} & Longitude value when placing a new fire at coordinates. Use for command \texttt{HSL/Fire\_On\_Coordinates}.\\ \hline
Double & \texttt{HSL/Fire/SetElevation} & Elevation over ground when placing a new fire at coordinates. Use for command \texttt{HSL/Fire\_On\_Coordinates}.\\ \hline

Integer & \texttt{HSL/Fire/UpdatePositions} & When 1: Recalculate the fire positions. XP world coordinate positioning is not precise when the aircraft is far away. Hence, it is advisable to re-calcuate the position when the aircraft gets closer.\\ \hline

Integer & \texttt{HSL/Fire/RemoveFires} & When 1, all fires are removed.\\ \hline
\end{tabularx}

\subsection{ReadOnly Values}


\subsubsection{Calculated Values}
\begin{tabularx}{\linewidth}{| l | l | X |} \hline
\textbf{Type} & \textbf{DataRef Name}& \textbf{Description}\\ \hline
Double & \texttt{HSL/Calculated/FrameTime} & Time between flight loops.\\ \hline
Double & \texttt{HSL/Calculated/NewRopeLength} &  Current rope length with expansion.\\ \hline
Double & \texttt{HSL/Calculated/RopeStretchRelative} &  Relative value of the expansion (0 = no expansion, 1 = twice the size)\\ \hline
Double & \texttt{HSL/Calculated/RopeForceScalar} & Scalar value of the rope force [N]\\ \hline
Double & \texttt{HSL/Calculated/RopeStretchSpeed} & Speed of the expanding rope [m/s]. Negative value for contracting rope.\\ \hline
Double & \texttt{HSL/Calculated/RopeCorrectedD} & Damping value for the current rope load = 2 * D * sqrt(m/K)\\ \hline
Float Array[3] & \texttt{HSL/Calculated/VectorHelicopterPosition} & Position of the winch in OpenGL coordinates.\\ \hline
Float Array[3] & \texttt{HSL/Calculated/VectorHookPosition} & Position of the end of the rope, no matter what's connected.\\ \hline
\end{tabularx}

\subsubsection{Rope}
\begin{tabularx}{\linewidth}{| l | l | X |} \hline
\textbf{Type} & \textbf{DataRef Name}& \textbf{Description}\\ \hline
Integer & \texttt{HSL/Rope/RopeRuptured} & 1 if the rope is ruptured, 0 otherwise.\\ \hline
\end{tabularx}

\subsubsection{Hook}
\begin{tabularx}{\linewidth}{| l | l | X |} \hline
\textbf{Type} & \textbf{DataRef Name}& \textbf{Description}\\ \hline
Integer & \texttt{HSL/Hook/Connected} & 1 if the hook is connected to the rope, 0 otherwise. Hook only disconnects on rope rupture.\\ \hline
Integer & \texttt{HSL/Hook/FollowOnly} & 1 if the cargo is connected, 0 otherwise.\\ \hline
Integer & \texttt{HSL/Hook/DrawingEnabled} & 1 if the hook is currently drawn, 0 otherwise.\\ \hline
Integer & \texttt{HSL/Hook/InstancedDrawing} & 1 if instanced drawing is enabled.\\ \hline
\end{tabularx}

\subsubsection{Cargo}
\begin{tabularx}{\linewidth}{| l | l | X |} \hline
\textbf{Type} & \textbf{DataRef Name}& \textbf{Description}\\ \hline
Float Array[3] & \texttt{HSL/Cargo/Position} & Position of the cargo.\\ \hline
Integer & \texttt{HSL/Cargo/Connected} & 1 if the cargo is connected to the rope, 0 otherwise.\\ \hline
Integer & \texttt{HSL/Cargo/FollowOnly} &  1 if the cargo is neither connected nor placed on terrain, 0 otherwise.\\ \hline
Integer & \texttt{HSL/Cargo/DrawingEnabled} & 1 if the cargo is drawn, 0 otherwise.\\ \hline
Integer & \texttt{HSL/Cargo/InstancedDrawing} & 1 if instanced drawing is enabled.\\ \hline
Double & \texttt{HSL/Cargo/BambiBucketWaterWeight} & Weight of the water in the bambi bucket [kg]\\ \hline
Double & \texttt{HSL/Cargo/BambiBucketWaterLevel} & Water level in the bambi bucket, between 0 and 1. This is computed, but may be written too.\\ \hline
\end{tabularx}

\subsubsection{Fire}
\begin{tabularx}{\linewidth}{| l | l | X |} \hline
\textbf{Type} & \textbf{DataRef Name}& \textbf{Description}\\ \hline
Integer & \texttt{HSL/Fire/CreateFailed} & Fire creation failed.\\ \hline
Double & \texttt{HSL/Fire/Count} & Number of active fires.\\ \hline
Float Array[10] & \texttt{HSL/Fire/FireStrength} & Strength of the currently active fires, in water weight [kg].\\ \hline
\end{tabularx}

\section{License}
This program is free software: you can redistribute it and/or modify it under the terms of the GNU General Public License as published by the Free Software Foundation, version 3.\\

This program is distributed in the hope that it will be useful, but WITHOUT ANY WARRANTY; without even the implied warranty of MERCHANTABILITY or FITNESS FOR A PARTICULAR PURPOSE. See the GNU General Public License for more details.\\

You should have received a copy of the GNU General Public License along with this program. If not, see <http://www.gnu.org/licenses/>.\\

\end{document} 